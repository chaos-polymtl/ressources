\documentclass[t, 11pt,xcolor=dvipsnames]{beamer}

%Insert number of slides
\beamertemplatenavigationsymbolsempty
\addtobeamertemplate{navigation symbols}{}{%
    \usebeamerfont{footline}%
    \usebeamercolor[fg]{footline}%
    \hspace{1em}%
    \insertframenumber/\inserttotalframenumber
}

% Redefine itemize
\def\labelitemi{--}


%Define colors useful for presentation
\definecolor{UniBlue}{RGB}{0,102,204}
\definecolor{UniOrange}{RGB}{255,128,0}
\definecolor{mygreen}{RGB}{120,190,33}
\newcommand{\green}[1]{\textcolor{ForestGreen}{#1}}

%\definecolor{color1}{HTML}{B3E2CD}
%\definecolor{color2}{HTML}{FDCDAC}
%\definecolor{color3}{HTML}{CBD5E8}
%\definecolor{color4}{HTML}{F4CAE4}
%\definecolor{color5}{HTML}{E6F5C9}

%\definecolor{color1}{HTML}{66C2A5}
%\definecolor{color2}{HTML}{FC8D62}
%\definecolor{color3}{HTML}{8DA0CB}
%\definecolor{color4}{HTML}{E78AC3}
%\definecolor{color5}{HTML}{A6D854}

\definecolor{color1}{HTML}{1B9E77}
\definecolor{color2}{HTML}{D95F02}
\definecolor{color3}{HTML}{7570B3}
\definecolor{color4}{HTML}{E7298A}
\definecolor{color5}{HTML}{66A61E}


\setbeamercolor{title}{fg=color1}
\setbeamercolor{frametitle}{fg=color1}
\setbeamercolor{structure}{fg=color3}
\setbeamercolor{footline}{fg=black}
\setbeamercolor{caption name}{fg=black}
\setbeamercolor{bibliography item}{fg=black}
\setbeamercolor*{bibliography entry title}{fg=black}
\setbeamercolor*{bibliography entry author}{fg=black}
\setbeamercolor*{bibliography entry location}{fg=black}
\setbeamercolor*{bibliography entry note}{fg=black}


\setbeamertemplate{section in toc}[sections numbered]
\setbeamertemplate{caption}[numbered]
\setbeamertemplate{bibliography item}{\insertbiblabel}

%Additional packages
\usepackage{blkarray}
\usepackage{amsmath}
\usepackage{amsfonts}
\usepackage{amssymb}
\usepackage{algorithm2e}
\usepackage{appendixnumberbeamer}
\usepackage{pgfplots}
\usepgfplotslibrary{groupplots,dateplot}
\usetikzlibrary{patterns,shapes.arrows}
\pgfplotsset{compat=newest}
\usepackage{multimedia}
\usepackage{media9}
\usepackage{pbox}
\usepackage{multirow}
\usepackage[makeroom]{cancel}
\usepackage[export]{adjustbox}
\usepackage{tabularx}
\usepackage{caption}
\usepackage{xcolor}
\usepackage{tikz,pgfplotstable,filecontents}
\usepackage{listings}
\usepackage{enumitem}
\usepackage{algorithmic}

%GANTT Stuff
\usepackage{pgfgantt}
\newgantttimeslotformat{stardate}{% \def\decomposestardate##1.##2\relax{%
\def\stardateyear{##1}\def\stardateday{##2}% }% \decomposestardate#1\relax%
\pgfcalendardatetojulian{\stardateyear-01-01}{#2}% \advance#2 by-1\relax%
\advance#2 by\stardateday\relax%
}


\newganttlinktype{rdldr*}{%
  \draw [/pgfgantt/link]
    (\xLeft, \yUpper) --
    (\xLeft + \ganttvalueof{link bulge 1} * \ganttvalueof{x unit},
      \yUpper) --
    ($(\xLeft + \ganttvalueof{link bulge 1} * \ganttvalueof{x unit},
      \yUpper)!%
      \ganttvalueof{link mid}!%
      (\xLeft + \ganttvalueof{link bulge 1} * \ganttvalueof{x unit},
      \yLower)$) --
    ($(\xRight - \ganttvalueof{link bulge 2} * \ganttvalueof{x unit},
      \yUpper)!%
      \ganttvalueof{link mid}!%
      (\xRight - \ganttvalueof{link bulge 2} * \ganttvalueof{x unit},
      \yLower)$) --
    (\xRight - \ganttvalueof{link bulge 2} * \ganttvalueof{x unit},
      \yLower) --
    (\xRight, \yLower);%
}
\ganttset{
  link bulge 1/.link=/pgfgantt/link bulge,
  link bulge 2/.link=/pgfgantt/link bulge}
  
\hyphenation{in-com-press-ible}
\hyphenation{com-press-ible}

\newcommand{\link}[3][blue]{\hyperlink{#2}{\color{#1}{#3}}}%



\newcommand\Dganttbar[4]{%
  \ganttbar{#1}{#3}{#4}\ganttbar[inline,bar label font=\footnotesize]{#2}{#3}{#4}
}
% End of GANTT stuff



%Shading text: useful to highlight information
\usepackage{framed, color}
\definecolor{shadecolor}{RGB}{220,220,220}
\usepackage{makecell}

\usepackage{subcaption}
%Define logos for subojectives
% \newcommand{\logoso1}{\setbeamertemplate{logo}{\includegraphics[width=0.1\textwidth]{images/so1.png}}
\def\mathunderline#1#2{\color{#1}\underline{{\color{black}#2}}\color{black}}


\usepackage{siunitx}

% Add an outline slide at the beginning of each new section
\AtBeginSection[]
{
	{
		\setbeamertemplate{footline}{} %this line removes slides numbers
		\begin{frame}[noframenumbering]
			\frametitle{Outline}
			\tableofcontents[currentsection]
		\end{frame}
	}
}

\title{\textbf{Supervising internships}}
\subtitle{Let's talk about it}
\author{Bruno \textit{problembär} Blais}


\begin{document}
    %% Title slide
\begin{frame}
\vspace{0.5cm}
	\begin{figure}
  		\includegraphics[scale = 0.12]{images/logo_lethe.png}\hspace*{0.2cm}
  		\includegraphics[scale = 0.4]{images/logo-poly.jpg}
	\end{figure}
	\titlepage
\end{frame}

%%Contents slide
\begin{frame}
	\frametitle{\textbf{Outline}}
	\tableofcontents
\end{frame}

%% Other slides
\section{Motivation behind this small training session}

\begin{frame} {Motivation}

	\visible<1->{
			\textbf{Internships are hit or miss}
			\vspace{1em}

			\textbf{\color{color2} Hit}
			\begin{itemize}
				\item Learning experience for students and for you
				\item Significantly advance your project
				\item Test different hypothesis
			\end{itemize}

			\textbf{\color{color3} Miss}
			\begin{itemize}
				\item Impression of time wasted
				\item Inadequate or non-reusable results
				\item Frustrations (on both sides)
			\end{itemize}
	}

	\visible<2->{
	\begin{shaded}
		\begin{itemize}
		\item \textbf{Good experience to share?}
		\item \textbf{Bad experience to share?}
		\end{itemize}
	\end{shaded}

	}

\end{frame}

\begin{frame} {This seminar}
\begin{block}{Slides}
	The LaTeX sources for the slides for this seminar are freely available on Github in the following public repository \url{https://github.com/lethe-cfd/ressources}	
\end{block}

\begin{block}{Internship}
One of the main element of this training is the focus on designing an internship plan. We will discuss an example of such plan later, but a template and an example are available on the Github repository.
\end{block}

\begin{block}{License}
The entire content of this training session is available under an MIT license. Feel free to reuse and adapt.
\end{block}

\end{frame}





\section{Planning the internship}

\begin{frame}[fragile]{Why?}

	Key steps to success. Plan and anticipate challenges. Internships are short, make the best use of everyone's time.
	
	\begin{block}{Key elements}
		\begin{itemize}
		\item When?
		\item Where?
		\item What?
		\item Who?
		\item Communication
		\item Time organization
		\item Budget
		\end{itemize}
	\end{block}
\end{frame}



\begin{frame}{When?}

	\begin{block}{Duration}
		Internships last between 8 weeks and 16 weeks.  Longer internships generally lead to better results, but they require more planning. 
	\end{block}

	\visible<2->
	{
	\begin{block}{Working schedule}
		\begin{itemize}
			\item What is your work schedule? 
			\item What do you expect will be the work schedule of your intern?
			\item How many hours per week do you expect them to work?
		\end{itemize}
		\textbf{Be explicit. Establish these requirements in written form.}
	\end{block}
	}

	\visible<3->
	{
	\begin{block}{To consider}
		\begin{itemize}
			\item Will you be taking vacations? 
			\item Will you be going on conferences? 
		\end{itemize}
	\end{block}
	}

	\visible<4->{
		\begin{shaded}
			\textbf{Anecdote}
		\end{shaded}
		}

\end{frame}

\begin{frame}{Where?}

	\visible<1->{
		\begin{shaded}
			\textbf{Another anecdote}
		\end{shaded}
		}

		\visible<2->
		{
	\textbf{Fun fact:} Your intern needs a desk. 
	
	\begin{itemize}
		\item Where should that desk be located?  \begin{itemize} \item Close? \item Far? \end{itemize}
		\item Is there a desk available? 
		\item Is the room where the desk is located locked?
	\end{itemize}
	}

	\visible<3->
	{
		\textbf{Fun fact:} Your intern may need access to a laboratory.
		\begin{itemize}
			\item What type of training do they need?
			\item Do they need a key? 
		\end{itemize}
	}
	
\end{frame}

\begin{frame}{What?}
	\visible<1->{
		\begin{shaded}
			\textbf{Describe to me what type of work you would like to give your intern}
		\end{shaded}
		}

	\visible<2->
	{
		I like to divide internships into three categories
		\begin{itemize}
			\item Companion
			\item Parallel work
			\item Orthogonal work
		\end{itemize}
	}

\end{frame}

\begin{frame}{Companion}

	\begin{block}{Description}
		Internship where the intern closely collaborates with you and works with you on a daily basis.
	\end{block}

	\textbf{Examples}
	\begin{itemize}
		\item Assemble an experimental setup with your intern
		\item Collect data with your intern 
		\item Write a software with your intern
	\end{itemize}

	\textbf{\color{color2} Hit}
	\begin{itemize}
		\item Easy to keep track of what the intern is doing
		\item Keep track of the quality of the results
		\item Good learning experience / social
	\end{itemize}

	\textbf{\color{color3} Miss}
	\begin{itemize}
		\item Can be unproductive
		\item Can be frustrating for the intern
	\end{itemize}

\end{frame}

\begin{frame}{Parallel work}

	\begin{block}{Description}
Internship where the intern works on a component which is key to your project
	\end{block}

	\textbf{Examples}
	\begin{itemize}
		\item Caracterize the material you are developing
		\item Develop a script to post-process simulations you are doing
		\item Run simulations of your case
	\end{itemize}

	\textbf{\color{color2} Hit}
	\begin{itemize}
		\item Accelerates your project 
		\item Good learning experience
	\end{itemize}

	\textbf{\color{color3} Miss}
	\begin{itemize}
		\item Quality needs to be monitored adequately.
		\item Training required which may take a lot of your time.
	\end{itemize}
\end{frame}

\begin{frame}{Orthogonal work}

	\begin{block}{Description}
Internship where the intern explores one of your side project
	\end{block}

	\textbf{Examples}
	\begin{itemize}
		\item Simulate a new test case which you don't need for your project
		\item Develop a new feature in your software
		\item Try another type of catalyst for an experience you are doing
	\end{itemize}

	\textbf{\color{color2} Hit}
	\begin{itemize}
		\item Can lead to completely novel research
		\item Good learning experience
	\end{itemize}

	\textbf{\color{color3} Miss}
	\begin{itemize}
		\item Quality needs to be monitored adequately.
		\item Training required which may take a lot of your time.
	\end{itemize}
\end{frame}


\begin{frame}{Who?}
You need to establish who is responsible for what regarding your intern.

\begin{block}{Role of your advisor}
	\begin{itemize}
	\item What is the role that your advisor will play in the internship?
	\item Who takes care of what? Do not assume, ask questions ahead of time.
	\item Will they meet the intern without you?
	\end{itemize}
\end{block}


\visible<2->{
	\begin{shaded}
		\textbf{How do you envision this in your case?}
	\end{shaded}
	}
\end{frame}


\begin{frame}{Communication and management}
	How will your intern communicate with you?

	\begin{block}{Before}
		E-mails? Slack?
	\end{block}

	\begin{block}{During}
		Daily meetings? Weekly meetings? No meetings?
	\end{block}

	\begin{block}{After}
		Why is this so important?
	\end{block}

\end{frame}


\begin{frame}{Organize the internship}

	\tikzset{every picture/.style={xscale=0.7,yscale=0.8,transform shape}}

	\begin{figure}[h]

		\begin{ganttchart}[
		group label node/.append style={align=left,text width=5em},
		bar label node/.append style={align=left,text width=5em},
		milestone label node/.append style={align=left,text width=5em},
		Mile1/.style={milestone/.append style={fill=black,align=left,text width=0.6em}},
		x unit=3.2mm,y unit chart=3.8mm,y unit title=4.3mm,vgrid={draw=none,dotted}, title height=1, bar height=.7, group height=0.4,
		]{1}{42}
		  \gantttitle{Week 2}{6} 
		  \gantttitle{Week 4}{6}
		  \gantttitle{Week 6}{6} 
		  \gantttitle{Week 8}{6}
		  \gantttitle{Week 10}{6}
		  \gantttitle{Week 12}{6}
		  \gantttitle{Week 14}{6}\\ 
		  \ganttgroup{\footnotesize Work item 1}{1}{38}\\
		  \ganttset{bar/.append style={fill=color1}}
		  \Dganttbar{\footnotesize Element 1}{\scriptsize {}}{1}{12}\\ 
		  \ganttbar{\footnotesize Element 2}{12}{24}\\ 
		  \ganttbar{\footnotesize Element 3}{24}{36}\\
		  \ganttbar{\footnotesize Element 4}{10}{20} % trajectories
		  \ganttset{bar/.append style={fill=color5}}
		  \Dganttbar{}{}{20}{30}\\

	  
		  \ganttgroup{\footnotesize WP 2}{3}{24} \\
		  \ganttset{bar/.append style={fill=color2}}
		  \Dganttbar{\footnotesize Element 1}{}{3}{6}\\
		  \Dganttbar{\footnotesize Element 2}{}{6}{20}\\
		  \ganttmilestone[Mile1]{\footnotesize Deliverable}{20} \\
	  
	  
		  \ganttgroup{\footnotesize WP 3}{3}{24} \\
		  \ganttset{bar/.append style={fill=color3}}
		  \Dganttbar{\footnotesize Element 1}{}{12}{30}\\
		  \Dganttbar{\footnotesize Element 2}{}{14}{24}\\
		  \ganttmilestone[Mile1]{\footnotesize Deliverable}{30} \\
	  
		  \ganttgroup{\footnotesize WP 3}{3}{24} \\
		  \ganttset{bar/.append style={fill=color4}}
		  \Dganttbar{\footnotesize Element 1}{}{3}{6}\\
		  \Dganttbar{\footnotesize Element 2}{}{6}{12}\\
		  \ganttmilestone[Mile1]{\footnotesize Deliverable}{20} \\
	  
		  \ganttlink[link/.style={-stealth,semithick,densely dashed}, link type=rdldr*,link bulge 1=0.5, link bulge 2=0.5,link mid=.937]{elem10}{elem4}
		  \ganttlink[link/.style={-stealth,semithick,densely dashed}, link type=rdldr*,link bulge 1=0.5, link bulge 2=0.5,link mid=.937]{elem24}{elem19}
		  \ganttlink[link/.style={-stealth,thick,densely dotted}, link type=rdldr*,link bulge 1=0.5, link bulge 2=0.5,link mid=.95]{elem23}{elem17}
		\end{ganttchart}
		  \label{fig:gantt}
	\end{figure}
\end{frame}


\section{Supervising the internship}

\begin{frame}{Discussion}

	\visible<1->{
		\begin{shaded}
			\textbf{Why?}

Why are you supervising and intern this summer?
		\end{shaded}
		}
	\visible<2->{
		\begin{shaded}
			\textbf{Challenges that may be faced?}

			Let's discuss together the challenges that you may face during your supervision of an internship.
		\end{shaded}
		}

	\visible<3->{
		\begin{shaded}
			\textbf{Potential outcomes?}

			Let's discuss together the potential outcomes of this internship
		\end{shaded}
		}



		
\end{frame}

\begin{frame}{Steps during the internship}
	\begin{itemize}
		\item Onboarding: Getting to know the lab, the space, etc.
		\item Learning and litterature: Getting to know the topic itself
		\item First deliverable: First contribution of your intern
		\item ...
		\item Final report (if needed by the institution or your lab)
	\end{itemize}

	\visible<2->{
	\begin{block}{Communication}
		Throughout the internship, remember that communication is key. Be explicit and explain clearly your expectactions from the internships.
	\end{block}
	}

	\visible<3->{
		\begin{block}{Be reasonable}
First and foremost, internships are a learning experience. Try to have reasonable expectations of your interns.
		\end{block}
		}
\end{frame}

\section{Following through}

\begin{frame}{Planning the \textit{after}}

\begin{block}{Leveraging outcomes}
	If you want to leverage the outcome of the research produced by your intern, you need to plan keeping track of the raw data and software they produce.
	
	\textbf{Plan this while the intern is still there!}
\end{block}

\begin{block}{Be fair}
If the work produced by intern plays a role within one of  your publications, make sure to include the intern as a co-author. This is good their career!
\end{block}


\end{frame}

\section{Conclusions}
\begin{frame}{Take home message}
\begin{itemize}
	\item Internship are an amazing experience. I encourage you to take them seriously.
	\item Can greatly help you.
	\item Enable you to learn more about yourself and develop your mentoring skills.
\end{itemize}

\textbf{Be careful:}
\begin{itemize}
	\item Be mindful of your intern.
	\item Supervising interns can be very challenging. It is always dependent on their skillsets and their motivations.
\end{itemize}

\textbf{Be inclusive:}
\begin{itemize}
	\item Eat lunch with your intern.
	\item Try to learn about where they area from, what are their ambitions.
	\item Tell them about yourself, do not keep them isolated from the big picture.
\end{itemize}
\end{frame}


\end{document}