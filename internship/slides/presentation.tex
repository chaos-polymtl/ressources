\documentclass[t, 11pt,xcolor=dvipsnames]{beamer}

%Insert number of slides
\beamertemplatenavigationsymbolsempty
\addtobeamertemplate{navigation symbols}{}{%
    \usebeamerfont{footline}%
    \usebeamercolor[fg]{footline}%
    \hspace{1em}%
    \insertframenumber/\inserttotalframenumber
}

% Redefine itemize
\def\labelitemi{--}


%Define colors useful for presentation
\definecolor{UniBlue}{RGB}{0,102,204}
\definecolor{UniOrange}{RGB}{255,128,0}
\definecolor{mygreen}{RGB}{120,190,33}
\newcommand{\green}[1]{\textcolor{ForestGreen}{#1}}

%\definecolor{color1}{HTML}{B3E2CD}
%\definecolor{color2}{HTML}{FDCDAC}
%\definecolor{color3}{HTML}{CBD5E8}
%\definecolor{color4}{HTML}{F4CAE4}
%\definecolor{color5}{HTML}{E6F5C9}

%\definecolor{color1}{HTML}{66C2A5}
%\definecolor{color2}{HTML}{FC8D62}
%\definecolor{color3}{HTML}{8DA0CB}
%\definecolor{color4}{HTML}{E78AC3}
%\definecolor{color5}{HTML}{A6D854}

\definecolor{color1}{HTML}{1B9E77}
\definecolor{color2}{HTML}{D95F02}
\definecolor{color3}{HTML}{7570B3}
\definecolor{color4}{HTML}{E7298A}
\definecolor{color5}{HTML}{66A61E}


\setbeamercolor{title}{fg=color1}
\setbeamercolor{frametitle}{fg=color1}
\setbeamercolor{structure}{fg=color3}
\setbeamercolor{footline}{fg=black}
\setbeamercolor{caption name}{fg=black}
\setbeamercolor{bibliography item}{fg=black}
\setbeamercolor*{bibliography entry title}{fg=black}
\setbeamercolor*{bibliography entry author}{fg=black}
\setbeamercolor*{bibliography entry location}{fg=black}
\setbeamercolor*{bibliography entry note}{fg=black}


\setbeamertemplate{section in toc}[sections numbered]
\setbeamertemplate{caption}[numbered]
\setbeamertemplate{bibliography item}{\insertbiblabel}

%Additional packages
\usepackage{blkarray}
\usepackage{amsmath}
\usepackage{amsfonts}
\usepackage{amssymb}
\usepackage{algorithm2e}
\usepackage{appendixnumberbeamer}
\usepackage{pgfplots}
\usepgfplotslibrary{groupplots,dateplot}
\usetikzlibrary{patterns,shapes.arrows}
\pgfplotsset{compat=newest}
\usepackage{multimedia}
\usepackage{media9}
\usepackage{pbox}
\usepackage{multirow}
\usepackage[makeroom]{cancel}
\usepackage[export]{adjustbox}
\usepackage{tabularx}
\usepackage{caption}
\usepackage{xcolor}
\usepackage{tikz,pgfplotstable,filecontents}
\usepackage{listings}
\usepackage{enumitem}
\usepackage{algorithmic}


%Shading text: useful to highlight information
\usepackage{framed, color}
\definecolor{shadecolor}{RGB}{220,220,220}
\usepackage{makecell}

\usepackage{subcaption}
%Define logos for subojectives
% \newcommand{\logoso1}{\setbeamertemplate{logo}{\includegraphics[width=0.1\textwidth]{images/so1.png}}
\def\mathunderline#1#2{\color{#1}\underline{{\color{black}#2}}\color{black}}


\usepackage{siunitx}

% Add an outline slide at the beginning of each new section
\AtBeginSection[]
{
	{
		\setbeamertemplate{footline}{} %this line removes slides numbers
		\begin{frame}[noframenumbering]
			\frametitle{Outline}
			\tableofcontents[currentsection]
		\end{frame}
	}
}

\title{\textbf{Supervising internships}}
\subtitle{Let's talk about it}
\author{Bruno \textit{problembär} Blais}


\begin{document}
    %% Title slide
\begin{frame}
\vspace{0.5cm}
	\begin{figure}
  		\includegraphics[scale = 0.12]{images/logo_lethe.png}\hspace*{0.2cm}
  		\includegraphics[scale = 0.4]{images/logo-poly.jpg}
	\end{figure}
	\titlepage
\end{frame}

%%Contents slide
\begin{frame}
	\frametitle{\textbf{Outline}}
	\tableofcontents
\end{frame}

%% Other slides
\section{Motivation behind this small training session}

\begin{frame} {Motivation}

	\visible<1->{
			\textbf{Internships are hit or miss}
			\vspace{1em}

			\textbf{\color{color2} Hit}
			\begin{itemize}
				\item Learning experience for students and for you
				\item Significantly advance your project
				\item Test different hypothesis
			\end{itemize}

			\textbf{\color{color3} Miss}
			\begin{itemize}
				\item Impression of time wasted
				\item Inadequate or non-reusable results
				\item Frustrations (on both sides)
			\end{itemize}
	}

	\visible<2->{
	\begin{shaded}
		\begin{itemize}
		\item \textbf{Good experience to share?}
		\item \textbf{Bad experience to share?}
		\end{itemize}
	\end{shaded}

	}

\end{frame}

\begin{frame} {This seminar}
\begin{block}{Slides}
	The LaTeX sources for the slides for this seminar are freely available on Github in the following public repository \url{https://github.com/lethe-cfd/ressources}	
\end{block}

\begin{block}{Internship}
One of the main element of this training is the focus on designing an internship plan. We will discuss an example of such plan later, but a template and an example are available on the Github repository.
\end{block}

\begin{block}{License}
The entire content of this training session is available under an MIT license. Feel free to reuse and adapt.
\end{block}

\end{frame}





\section{Planning the internship}

\begin{frame}[fragile]{Why?}

	Key steps to success. Plan and anticipate challenges. Internships are short, make the best use of your time.
	
	\begin{block}{Key elements}
		\begin{itemize}
		\item When?
		\item Where?
		\item What?
		\item Communication
		\item Work ethics
		\item Time organization
		\item Budget
		\end{itemize}
	\end{block}
\end{frame}

\begin{frame}{When?}

	\begin{block}{Duration}
		Internships last between 8 weeks and 16 weeks.  Longer internships generally lead to better results, but they require more planning. 
	\end{block}

	\visible<2->
	{
	\begin{block}{Working schedule}
		\begin{itemize}
			\item What is your work schedule? 
			\item What do you expect will be the work schedule of your intern?
			\item How many hours per week do you expect them to work?
		\end{itemize}
	\end{block}
	}

	\visible<3->
	{
	\begin{block}{To consider}
		\begin{itemize}
			\item Will you be taking vacations? 
			\item Will you be going on conferences? 
		\end{itemize}
	\end{block}
	}

	\visible<4->{
		\begin{shaded}
			\textbf{Anecdote}
		\end{shaded}
		}

\end{frame}

\begin{frame}{Where?}

	\visible<1->{
		\begin{shaded}
			\textbf{Another anecdote}
		\end{shaded}
		}

		\visible<2->
		{
	\textbf{Fun fact:} Your intern needs a desk. 
	
	\begin{itemize}
		\item Where should that desk be located?  \begin{itemize} \item Close? \item Far? \end{itemize}
		\item Is there a desk available? 
		\item Is the room where the desk is located locked?
	\end{itemize}
	}

	\visible<3->
	{
		\textbf{Fun fact:} Your intern may need access to a laboratory.
		\begin{itemize}
			\item What type of training do they need?
			\item Do they need a key? 
		\end{itemize}
	}
	
\end{frame}

\begin{frame}{What?}
	\visible<1->{
		\begin{shaded}
			\textbf{Describe to me what type of work you would like to give your intern}
		\end{shaded}
		}

	\visible<2->
	{
		I like to divide internships into three categories
		\begin{itemize}
			\item Companion
			\item Parallel work
			\item Orthogonal work
		\end{itemize}
	}

\end{frame}

\begin{frame}{Companion}

	\begin{block}{Description}
		Inernship where the intern closely collaborates with you and works with you on a daily basis.
	\end{block}

	\textbf{Examples}
	\begin{itemize}
		\item Assemble an experimental setup with your intern
		\item Collect data with your intern 
		\item Write a software with your intern (either co-coding or splitting small routines)
	\end{itemize}

\end{frame}



\section{Supervising the internship}

\section{Following through}

	
%\begin{framed}
%	Find a scalar $T$ in $\Omega \rightarrow \mathbb{R}$ satisfying:
%	\begin{align*}
%			-\partial_j \partial_j T =& 10 \ \textrm{ in } \Omega\\
%			 T =& 0 \ \textrm{ in } \Gamma
%	\end{align*}
%	\end{framed}
%
%	\visible<1->{
%		\begin{shaded}
%			\textbf{Goal of all iterative methods}
%			\begin{itemize}
%			\item Goal is to annhilate some components of teh residual vector $r=b-\mathcal{A}x$
%			\item We will try to find a solution for $x$ that satisfies $\lVert x \rVert < \mathrm{tol}$
%			\end{itemize}
%	\end{shaded}}
\end{document}